\documentclass[11pt]{article}
\usepackage{amsmath, amssymb, amscd, amsthm, amsfonts}
\usepackage{graphicx}
\usepackage{hyperref}
\usepackage{acronym}
\usepackage[utf8]{inputenc}
\usepackage{glossaries}

\oddsidemargin 0pt
\evensidemargin 0pt
\marginparwidth 40pt
\marginparsep 10pt
\topmargin -20pt
\headsep 10pt
\textheight 8.7in
\textwidth 6.65in
\linespread{1.2}

% for centering the title page
\usepackage{titling}
\renewcommand\maketitlehooka{\null\mbox{}\vfill}
\renewcommand\maketitlehookd{\vfill\null}

% ------- glossary -------
% to build the glossary when changed run the build make_glossaries.cmd file first
\makeglossaries

\newglossaryentry{computer}
{
  name=computer,
  description={is a programmable machine that receives input,
               stores and manipulates data, and provides
               output in a useful format}
}
\newacronym{gcd}{GCD}{Greatest Common Divisor}

\newacronym{lcm}{LCM}{Least Common Multiple}
% ------- glossary -------

\title{Seminararbeit Red-Black Trees}
\author{Yannik Höll}
\date{\today}

\begin{document}

\begin{titlingpage}
    \maketitle
\end{titlingpage}
\pagebreak

\tableofcontents
\pagebreak

\glsaddall
\printglossary 
\pagebreak

\section{Einleitung}

The \Gls{computer}
Spass

\section{Zugriffsprimitiven}

\section{Implementierungsvarianten}

\section{Benchmarks}

\section{Speicher- und Zeitkomplexität}

\section{Verwendung}

\pagebreak
\bibliographystyle{alpha}
\bibliography{references} % see references.bib for bibliography management

\end{document}

% for citation:
% \cite{aad} for simple citation
% \cite[S. 20]{aad} for citation with page number(s)


% for acronyms:
% \acp{KDE} % K Desktop Environments (KDEs)
% \acsp{KDE} % KDEs
% \acfp{KDE} % K Desktop Environments (KDEs)
% \aclp{KDE} % K Desktop Environments
